\documentclass[]{article}

% 快點寫功課!!!
% \documentclass[]{article}
\makeatletter\if@twocolumn\PassOptionsToPackage{switch}{lineno}\else\fi\makeatother

  
\usepackage{amsmath,amsfonts,amsbsy,amssymb,tabulary,graphicx,times,caption,fancyhdr}
\usepackage[utf8]{inputenc}
\usepackage[paperheight=10in,paperwidth=6.5in,margin=2cm,headsep=.5cm,top=2.5cm,headheight=1cm]{geometry}
\renewenvironment{abstract} {\vspace*{-1pc}\trivlist\item[]\leftskip\oupIndent\hrulefill\par\vskip4pt\noindent\textbf{\abstractname}\mbox{\null}\\\relax}{\par\noindent\hrulefill\endtrivlist} 
\linespread{1.13} \date{}
\captionsetup[figure]{labelfont=sc,skip=1.4pt,aboveskip=1pc}
\captionsetup[table]{labelfont=sc,skip=1.4pt,labelsep=newline}



\makeatletter\def\oupIndent{1pt}
\def\author#1{\gdef\@author{\hskip-\dimexpr(\tabcolsep)\hskip\oupIndent\parbox{\dimexpr\textwidth-\oupIndent}{\centering\bfseries#1}}}
\def\title#1{\gdef\@title{\centering\bfseries\ifx\@articleType\@empty\else\@articleType\\\fi#1}}
\let\@articleType\@empty \def\articletype#1{\gdef\@articleType{{\normalfont\itshape#1}}}
\fancypagestyle{headings}{\fancyhf{}\fancyhead[C]{\RunningHead}\fancyhead[R]{\thepage}}\pagestyle{headings}
\makeatother

  


\tolerance=5000
%%%%%%%%%%%%%%%%%%%%%%%%%%%%%%%%%%%%%%%%%%%%%%%%%%%%%%%%%%%%%%%%%%%%%%%%%%
% Following additional macros are required to function some 
% functions which are not available in the class used.
%%%%%%%%%%%%%%%%%%%%%%%%%%%%%%%%%%%%%%%%%%%%%%%%%%%%%%%%%%%%%%%%%%%%%%%%%%
\usepackage{
  url,
multirow,morefloats,floatflt,cancel,tfrupee}
% \usepackage[hyphens]{url}
\usepackage{hyperref} 



\makeatletter


\AtBeginDocument{\@ifpackageloaded{textcomp}{}{\usepackage{textcomp}}}
\makeatother
\usepackage{colortbl}
\usepackage{xcolor}
\usepackage{pifont}
\usepackage[nointegrals]{wasysym}
\usepackage{enumitem}

\urlstyle{rm}
\makeatletter

%%%For Table column width calculation.
\def\mcWidth#1{\csname TY@F#1\endcsname+\tabcolsep}

%%Hacking center and right align for table
\def\cAlignHack{\rightskip\@flushglue\leftskip\@flushglue\parindent\z@\parfillskip\z@skip}
\def\rAlignHack{\rightskip\z@skip\leftskip\@flushglue \parindent\z@\parfillskip\z@skip}

%Etal definition in references
\@ifundefined{etal}{\def\etal{\textit{et~al}}}{}


%\if@twocolumn\usepackage{dblfloatfix}\fi
\usepackage{ifxetex}
\ifxetex\else\if@twocolumn\@ifpackageloaded{stfloats}{}{\usepackage{dblfloatfix}}\fi\fi

\AtBeginDocument{
\expandafter\ifx\csname eqalign\endcsname\relax
\def\eqalign#1{\null\vcenter{\def\\{\cr}\openup\jot\m@th
  \ialign{\strut$\displaystyle{##}$\hfil&$\displaystyle{{}##}$\hfil
      \crcr#1\crcr}}\,}
\fi
}

%For fixing hardfail when unicode letters appear inside table with endfloat
\AtBeginDocument{%
  \@ifpackageloaded{endfloat}%
   {\renewcommand\efloat@iwrite[1]{\immediate\expandafter\protected@write\csname efloat@post#1\endcsname{}}}{\newif\ifefloat@tables}%
}%

\def\BreakURLText#1{\@tfor\brk@tempa:=#1\do{\brk@tempa\hskip0pt}}
\let\lt=<
\let\gt=>
\def\processVert{\ifmmode|\else\textbar\fi}
\let\processvert\processVert

\@ifundefined{subparagraph}{
\def\subparagraph{\@startsection{paragraph}{5}{2\parindent}{0ex plus 0.1ex minus 0.1ex}%
{0ex}{\normalfont\small\itshape}}%
}{}

% These are now gobbled, so won't appear in the PDF.
\newcommand\role[1]{\unskip}
\newcommand\aucollab[1]{\unskip}
  
\@ifundefined{tsGraphicsScaleX}{\gdef\tsGraphicsScaleX{1}}{}
\@ifundefined{tsGraphicsScaleY}{\gdef\tsGraphicsScaleY{.9}}{}
% To automatically resize figures to fit inside the text area
\def\checkGraphicsWidth{\ifdim\Gin@nat@width>\linewidth
	\tsGraphicsScaleX\linewidth\else\Gin@nat@width\fi}

\def\checkGraphicsHeight{\ifdim\Gin@nat@height>.9\textheight
	\tsGraphicsScaleY\textheight\else\Gin@nat@height\fi}

\def\fixFloatSize#1{}%\@ifundefined{processdelayedfloats}{\setbox0=\hbox{\includegraphics{#1}}\ifnum\wd0<\columnwidth\relax\renewenvironment{figure*}{\begin{figure}}{\end{figure}}\fi}{}}
\let\ts@includegraphics\includegraphics

\def\inlinegraphic[#1]#2{{\edef\@tempa{#1}\edef\baseline@shift{\ifx\@tempa\@empty0\else#1\fi}\edef\tempZ{\the\numexpr(\numexpr(\baseline@shift*\f@size/100))}\protect\raisebox{\tempZ pt}{\ts@includegraphics{#2}}}}

%\renewcommand{\includegraphics}[1]{\ts@includegraphics[width=\checkGraphicsWidth]{#1}}
\AtBeginDocument{\def\includegraphics{\@ifnextchar{\ts@includegraphics}{\ts@includegraphics[width=\checkGraphicsWidth,height=\checkGraphicsHeight,keepaspectratio]}}}

\DeclareMathAlphabet{\mathpzc}{OT1}{pzc}{m}{it}

\def\URL#1#2{\@ifundefined{href}{#2}{\href{#1}{#2}}}


\PassOptionsToPackage{hyphens,spaces,obeyspaces}{url}

% \usepackage{hyphens,spaces,obeyspaces}{url}
\def\UrlBreaks{\do\/\do-}
\usepackage{xurl}
\usepackage{hyperref}
\hypersetup{breaklinks=true}
% \makeatletter

%For url break

\def\UrlOrds{\do\*\do\-\do\~\do\'\do\"\do\-\do\/}%
% \g@addto@macro{\UrlBreaks}{\UrlOrds}


\def\UrlAlphabet{%
      \do\a\do\b\do\c\do\d\do\e\do\f\do\g\do\h\do\i\do\j%
      \do\k\do\l\do\m\do\n\do\o\do\p\do\q\do\r\do\s\do\t%
      \do\u\do\v\do\w\do\x\do\y\do\z\do\A\do\B\do\C\do\D%
      \do\E\do\F\do\G\do\H\do\I\do\J\do\K\do\L\do\M\do\N%
      \do\O\do\P\do\Q\do\R\do\S\do\T\do\U\do\V\do\W\do\X%
      \do\Y\do\Z}
\def\UrlDigits{\do\1\do\2\do\3\do\4\do\5\do\6\do\7\do\8\do\9\do\0}
\g@addto@macro{\UrlBreaks}{\UrlOrds}
\g@addto@macro{\UrlBreaks}{\UrlAlphabet}
\g@addto@macro{\UrlBreaks}{\UrlDigits}
% \makeatother


\edef\fntEncoding{\f@encoding}
\def\EUoneEnc{EU1}
\makeatother
\def\floatpagefraction{0.8} 
\def\dblfloatpagefraction{0.8}
\def\style#1#2{#2}
\def\xxxguillemotleft{\fontencoding{T1}\selectfont\guillemotleft}
\def\xxxguillemotright{\fontencoding{T1}\selectfont\guillemotright}

\newif\ifmultipleabstract\multipleabstractfalse%
\newenvironment{typesetAbstractGroup}{}{}%

%%%%%%%%%%%%%%%%%%%%%%%%%%%%%%%%%%%%%%%%%%%%%%%%%%%%%%%%%%%%%%%%%%%%%%%%%%



% \usepackage[numbers,sort&compress]{natbib}

\usepackage[sorting=none, citestyle=verbose-inote,backref=true,ibidtracker=context,mincrossrefs=99,backend=biber, 
url = false,
doi = false, isbn=false,]{biblatex}

% \usepackage[style=verbose-inote,backref=true,ibidtracker=context]{biblatex}


\usepackage{xurl}

% \usepackage[style=verbose-ibid]{biblatex}
\AtEveryCitekey{\clearfield{url}}
\AtEveryCitekey{\clearfield{howpublished}} 


% If you want to break on URL numbers
\setcounter{biburlnumpenalty}{9000}
% If you want to break on URL lower case letters
\setcounter{biburllcpenalty}{9000}
% If you want to break on URL UPPER CASE letters
\setcounter{biburlucpenalty}{9000}

\biburlnumskip=0mu plus 1mu\relax
\biburlucskip=0mu plus 1mu\relax
\biburllcskip=0mu plus 1mu\relax

% Document
% https://www.emse.fr/~picard/files/biblatex.pdf
% Cheatsheet
% https://ftp.ntou.edu.tw/ctan/info/biblatex-cheatsheet/biblatex-cheatsheet.pdf

\addbibresource{R10A21126.bib}

\usepackage{lipsum} 
\usepackage[version=4]{mhchem}

\usepackage[T1]{fontenc}

%%%%%%%%%%%%%%%%%%%%%%%%%%%%%%%%%%%%%%%%%%
% Feature enabled:
%full-reference: true
%toc: yes
%%%%%%%%%%%%%%%%%%%%%%%%%%%%%%%%%%%%%%%%%%
\makeatletter\@ifundefined{tableofcontents}{\usepackage{typeset-custom-toc}}{}\makeatother
\usepackage{etoolbox}

% defines the paragraph spacing
\setlength{\parskip}{0.5em}




\begin{document}
% \sloppy



% \nocite{*}

\title{Air Pollution and Climate Change:\\ The Linkages and Implications for Policy}
\author{\textbf{\fontsize{14pt}{16.4pt}\selectfont{YIFAN WANG}}~\\\normalsize\normalfont {College of  Law \unskip, National Taiwan University}~\\{\normalsize\normalfont  E-mail: R10A21126@ntu.edu.tw}}
\def\RunningHead{{Air Pollution and Climate Change}}

\maketitle 


\begin{abstract}


  This study examines the correlation between air pollution and climate change by analyzing existing literature. It also tries to sort out the interaction and co-benefit of regulations targeting the two environmental issues. Furthermore, the paper presents a perspective on future environmental regulation policies.

%   This paper 
%   reviews the literature and identifies the relationship between air pollution and climate change. 
%   The paper also tries to sort out the interaction and co-benefit between air pollution regulation and climate change Regulatory actions and initiatives.
% In addition, the paper provides an outlook on the coming development of environmental regulation policies.

  % In addition, reasonable explanations for the expected solutions is be provided.

% This paper reviews the literature and identifies the relationship between the tax system and state sovereignty, especially in the age of globalization.
% The paper also tries to sort out the interaction between taxation and international commercial activities.
% In addition, the paper provides an outlook on the coming development of the international tax system.
  
% It calls for a multi-level and multilateral international tax regime that provides a friendly and stable environment for globalized economic transactions, contributes to net-zero carbon emissions, and maintains the tax base for governments around the world to achieve tax justice.

\end{abstract}

\def\keywordstitle{Keywords}

\smallskip\noindent\textbf{Keywords: }{Climate Change, Air Pollution, Co-benefit, Environmental Policy, Environmental Regulation} 


% \clearpage
\setcounter{tocdepth}{1}

\tableofcontents


\pagebreak

% Introduction
\section{Introduction}
\label{sec:intro}

Air pollution and climate change are two of the most pressing environmental issues that we face today\footcites{WorldBank}. 

Air pollution is the contamination of the indoor or outdoor environment by any chemical, physical or biological agent that modifies the natural characteristics of the atmosphere\footcite{WHO}.
Pollutants usually pose significant threats to public health, causing respiratory and other illnesses that can lead to morbidity and mortality. Air pollution has been proven to be one of the greatest environmental risks to health.
In 2019, 99\% of the world’s population was living in places where the WHO air quality guidelines levels were not met. The combined effects of ambient air pollution and household air pollution are associated with 6.7 million premature deaths annually\footcite{WHOAirOutdoor}.

Climate change, on the other hand, refers to the long-term changes in temperature, precipitation, and other weather patterns that are caused by human activities, particularly the emission of greenhouse gases (GHGs, such as carbon dioxide) into the atmosphere. These greenhouse gases trap heat and cause the Earth's temperature to rise, leading to global warming. Climate change can cause extreme weather events, such as hurricanes, floods, and droughts, and can also have devastating effects on human health, agriculture, and ecosystems\footcites{WorldBankClimateChange}.

Moreover, air quality is closely linked to the earth's climate and ecosystems worldwide, and the combustion of fossil fuels that drives air pollution also contributes to GHGs emissions\footcite{WHO}. 

Nowadays, GHGs are sometimes considered to be air pollutants, because they indirectly impact, change and hinder the living environment. For example, in 2012, Taiwan's Environmental Protection Administration Released an Announcement that GHGs are air pollutants\footcite{EPA2012}.

Therefore, it is necessary to put the two most important issues together and examine the correlation between them, and find efficient regulation policies.


% Therefore, reducing air pollution through policies presents a mutually beneficial approach that can lower the burden of disease associated with air pollution while mitigating climate change in the short and long term\footcite{WorldBank}.


% Air pollution and climate change are two of the most pressing environmental issues facing the world today. Air pollution is caused by a variety of sources, including industry, transportation, and agriculture. Climate change is caused by the release of greenhouse gases, such as carbon dioxide, into the atmosphere. In recent years, there has been growing concern about the impact of air pollution on the climate, and the need to reduce air pollution to mitigate the effects of climate change. In this study, we examine the correlation between air pollution and climate change, and propose solutions to reduce the impact of air pollution on the climate.


% Air pollution c




\section{Relationship between Air Pollution and Climate Change}
% \label{sec:Correlation}
\subsection{Commonalities}
\subsubsection{Sources}

Air pollutants are found in the atmosphere as gases, aerosols and particles. 
The main substances emitted include oxides of sulfur (SOx) from coal and heavy oil, nitrogen (NOx) from combustion and ammonia from agriculture and volatile organic compounds (VOCs) including from mining, refueling and industrial processes. Particles of black carbon and organic carbon from partially burnt fossil fuels and biomass are also important. Air pollutants can react chemically with each other to form secondary pollutants such as low-level ozone, itself a GHG. They can also slightly affect the concentration of long-lived GHGs such as carbon dioxide and methane\footcite{Warrilow2021}.

Sources of air pollution, such as the combustion of fossil fuels, are also sources of greenhouse gas emissions.
Burning fossil fuels for transportation, electricity, and heat generation represents the most significant contributor to human-generated greenhouse gas emissions. As per the United States Environmental Protection Agency (EPA), this activity is the primary origin of greenhouse gas emissions resulting from human activities in the United States\footcite{USEPA2015}. Worldwide, 73.2\% of greenhouse gas emissions stem from energy consumption for electricity, heat, and transport purposes\footcite{OurWorld2020}.


\subsubsection{Negative effects}

Both air pollutants and GHGs contribute to climate change, which can have severe consequences such as rising temperatures, melting of polar ice caps, rising sea levels, and extreme weather events.

Air pollutants and GHGs are both harmful to the environment. Air pollutants can cause damage by leading to acid rain, ozone depletion, and reduced crop yields, which can have ripple effects on ecosystems, wildlife, and food security. On the other hand, GHGs can indirectly harm the environment by contributing to rising sea levels and the development of extreme weather patterns.

\subsection{Differences}


\subsubsection{Lifetime}

 Most air pollutants have a short lifetime in the atmosphere of days or weeks, and their concentration is primarily determined by current emissions. This implies that if emissions were to halt, the majority of their immediate impact would diminish within weeks.

 In contrast, most greenhouse gases (GHGs) have a long atmospheric lifetime (decades to centuries), which means that they will remain at elevated levels for centuries, even if all emissions were to be dramatically reduced\footcite{Warrilow2021}.

 \subsubsection{Impacts}


 While air pollutants affect the climate at both the global scale and more strongly close to sources of pollution\footcite{Warrilow2021}, GHGs have a more far-reaching global impact. Due to the long lifetime of GHGs and the global intermixing of the Earth's atmosphere, countries are responsible for exporting a significant portion of the damage caused by their GHG emissions. However, the degree to which this results in inequality between GHG emitters and those affected by the resulting climate change is determined by the distribution of climate vulnerability. There is enormous global inequality, which means that countries with high emissions are putting the burden of climate change on others.\footcite{Althor2016}.
 


% Results and discussion
\section{Solutions}



\subsection{Co-benefit}

It is recognized that there is a co-benefit of reducing air pollution and mitigating climate change\footcite{WorldBank}. Many of the actions taken to achieve one goal can also help achieve the other. Climate policies targeting \ce{CO2} emissions from fossil fuels can simultaneously reduce emissions of air pollutants and their precursors, thus mitigating air pollution and associated health impacts\footcite{Li2019}.
For example, transitioning to clean energy sources such as wind, solar, and hydropower can help reduce both air pollution and greenhouse gas emissions. Promoting energy efficiency in buildings and transportation can also help reduce emissions of both air pollutants and greenhouse gases. Additionally, many of these actions can have positive impacts on public health, such as reducing respiratory illnesses associated with air pollution. By pursuing actions that achieve co-benefits, we can make progress toward addressing both air pollution and climate change more efficiently and effectively.
Quick action on reducing highly potent, short-lived climate pollutants - methane, tropospheric ozone, hydrofluorocarbons and black carbon—can significantly decrease the chances of triggering dangerous climate tipping points, like the irreversible release of carbon dioxide and methane from thawing Arctic permafrost\footcite{UN2019}.
\subsection{Decarbonize the economy}

Switching to renewable sources of energy is an important part of the solution to both climate change and air pollution. Promoting renewable energy sources, and improving energy efficiency and conservation can definitely help to reduce the emissions of air pollutants and GHGs.
The health benefits of reducing emissions from the burning of fossil fuels can occur in the near term. However, the reduction of carbon dioxide in the atmosphere would occur over a longer timeframe. If decarbonization efforts pay attention to non-\ce{CO2} pollutants as well, notably PM2.5, we cannot only expect better air quality but also health benefits in the short term\footcite{WorldBank}.


\subsection{Integrating the Regulations}

Since there is a co-benefit of reducing air pollution and GHGs, integrating the legislation of air pollution and climate change regulations is necessary both for international organizations and each country. 
A global accountability system is needed to eliminate climate inequity by specifying the responsibilities of the countries with high emissions of GHGs.

% It is time that this persistent and worsening climate inequity is resolved and for the largest emitting countries to act on their commitment of common but differentiated responsibilities.


% Conclusions
\section{Conclusions}
\label{sec:conclusions}
To conclude, the paper identifies the differences and commonalities between air pollution and climate change. There is the co-benefit of reducing air pollution and mitigating climate change, which means actions taken to achieve one goal can also help achieve the other. We recommend the implementation of policies to reduce air pollutants and greenhouse gas emissions, promote renewable energy sources, and improve energy efficiency and conservation. 
Since immediate changes in air pollution levels also have immediate impacts, steps should be taken to reduce emissions right away. 


\pagebreak


\sloppy

\printbibliography
  % \end{refcontext}
  % \printbibheading
  % \printbibliography[type=book,heading=subbibliography,title={Book Sources}]
  % \printbibliography[nottype=book,heading=subbibliography,title={Other Sources}]

\end{document}
